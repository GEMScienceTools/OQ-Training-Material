% -----------------------------------------------------------------------------
\section{Openquake demos}
%
OpenQuake usage help can be obtained by simply typing
\begin{Verbatim}[frame=single, commandchars=\\\{\}, fontsize=\small]
openquake@ubuntu:/media/vbox$ openquake 
\end{Verbatim}
and pressing \texttt{<ENTER>}. The provided info will be 
\begin{Verbatim}[frame=single, commandchars=\\\{\}, fontsize=\small]
usage: openquake [-h] [--version] [--force-inputs] [--config-file CONFIG_FILE]
                 [--output-type {db,xml}]
                 [--log-level {debug,info,warn,error,critical}]
                 [--log-file LOG_FILE] [--list-calculations]
                 [--list-outputs CALCULATION_ID]
                 [--export OUTPUT_ID TARGET_DIR] 
\end{Verbatim}
The main execution of OpenQuake is undertaken via the following 
command-line instruction (note that from this point on we'll indicate the 
the command prompt with a simple \texttt{\$}.

An OpenQuake analysis can be launched with the following command
\begin{Verbatim}[frame=single, commandchars=\\\{\}, fontsize=\small]
$ openquake --\textcolor{red}{config_file}=/PATH/TO/CONFIG/FILE --\textcolor{red}{output_type}=xml
\end{Verbatim}

\subsection{Area source demo hazard}
%
The input files for this demos can be found in the folder 
\texttt{area\_source\_demo\_hazard}

The input information consists of a PSHA input model for South East Asia 
developed in the context of the GSHAP project and containing 

\cleardoublepage

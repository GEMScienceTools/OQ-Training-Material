\documentclass[11pt,a4paper,headings=small,dvips]{scrbook}

\input{configuration.tex}
\setcounter{secnumdepth}{2}
\setcounter{tocdepth}{2}

\usepackage{xcolor}
\usepackage{framed}

\newenvironment{myfancybox}{%
  \def\FrameCommand{\fboxsep=\FrameSep \fcolorbox{blue01}{honeydew}}%
  \color{black}\MakeFramed {\FrameRestore}}%
 {\endMakeFramed}

\setlength{\parskip}{2.5mm}
\setlength{\parindent}{0.0mm}

\begin{document}

\setcounter{page}{1}

\begin{titlepage}
	\title{ \textcolor{blue01}{\textsf{\bfseries\Huge 
        Openquake's Training Workshop}}}
	\date{June 2012}
	\publishers{GEM Foundation, Pavia}
\end{titlepage}

\pagestyle{scrheadings}
\maketitle
\renewcommand*\thesection{\arabic{section}}
\renewcommand*\thefigure{\thesection.\arabic{figure}}
\clearpage
% -----------------------------------------------------------------------------
% -----------------------------------------------------------------------------
\chapter*{Introduction}
\cleardoublepage
% -----------------------------------------------------------------------------
% -----------------------------------------------------------------------------
\tableofcontents
\cleardoublepage
% -----------------------------------------------------------------------------
% -----------------------------------------------------------------------------
\chapter{Introduction to Openquake}
\begin{myfancybox}
The objectives of this chapter are:
\begin{itemize}
    \item aa
    \item bb
\end{itemize}
\end{myfancybox}
% -----------------------------------------------------------------------------
\section{Source typologies}
% -----------------------------------------------------------------------------
\section{Calculation workflows}
\cleardoublepage
% -----------------------------------------------------------------------------
% -----------------------------------------------------------------------------
\chapter{Using Openquake}
\begin{myfancybox}
The objectives of this chapter are:
\begin{itemize}
    \item aa
    \item bb
\end{itemize}
\end{myfancybox}
% -----------------------------------------------------------------------------
\section{Structure of an input model}
% -----------------------------------------------------------------------------
\section{Openquake demos}
\cleardoublepage
% -----------------------------------------------------------------------------
% -----------------------------------------------------------------------------
% -----------------------------------------------------------------------------
% -----------------------------------------------------------------------------
\chapter{Sub-saharan Africa demo}
\begin{myfancybox}
The objectives of this chapter are:
\begin{itemize}
    \item Understand the basic input and output file structure
    \item Learn how to compute hazard curves and hazard maps
\end{itemize}
\end{myfancybox}

\section{Model description}
The model used for this demo correponds to the model proposed by 
\citet{midzi1999} in the context of the GSHAP project.

This seismic source model contains 21 area sources distributed within a wide 
band that goes from the Red Sea until South Africa.

\section{Assignment: hazard calculation using the demo model}
The first assignment focuses on the calculation of hazard for a small area 
using Openquake and demo model provided.

\subsection{Analysis of results}
\cleardoublepage
% -----------------------------------------------------------------------------
% -----------------------------------------------------------------------------
\chapter{Openquake-Modeller}
\begin{myfancybox}
The objectives of this chapter are:
\begin{itemize}
    \item aa
    \item bb
\end{itemize}
\end{myfancybox}
\cleardoublepage
% -----------------------------------------------------------------------------
% -----------------------------------------------------------------------------
\chapter{Improving the Sub-Saharan Africa demo model}
\begin{myfancybox}
The objectives of this chapter are:
\begin{itemize}
    \item aa
    \item bb
\end{itemize}
\end{myfancybox}
\section{Catalogue processing}
\section{Calculation of activity rates}
\section{Assignment: Preparation of an updated OQ input demo model}
\subsection{Analysis of results}
\subsection{Comparison with previous results}
% -----------------------------------------------------------------------------
% -----------------------------------------------------------------------------
\cleardoublepage
\bibliographystyle{apalike}
\bibliography{./Bibliography/hazard}
\cleardoublepage
% -----------------------------------------------------------------------------
% -----------------------------------------------------------------------------
\end{document}

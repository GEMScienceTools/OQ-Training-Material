\section{Model description}

The model used for this demo corresponds to the model proposed by 
\citet{midzi1999} in the context of the GSHAP project.

This seismic source model contains 21 area sources distributed within a wide 
band that goes from the Red Sea until South Africa.

\section{Assignment: hazard calculation using the demo model}
The first assignment focuses on the calculation of hazard for a small area 
using Openquake and demo model provided.

\subsection{Analysis of results}


\section{A brief demo of the Modeller's Toolkit - An African example}

The following will outline how to use the Modeller's Toolkit for the above example. In this exercise we shall be using the GSHAP area sources used in the previous demonstration, and utilising a new earthquake catalogue to recalculate the activity rates for the zones. This same process could be applied to any earthquake catalogue for the region and any area source zone model. 

\subsection{The Earthquake Catalogues}

Two earthquake catalogues are used for this exercise:
\begin{enumerate}
\item The Global CMT database (1976 - 2011) (Figure \ref{fig:Subsahara_Catalogue_GCMT_1})
\item A ''Homogenised'' Instrumental Catalogue - Derived from the ISC bulletin, converting the magnitudes from different agencies according to a hierarchical selection (Figure \ref{fig:Subsahara_Catalogue_ISC_1})
\end{enumerate}
\textbf{The ''Homogenised'' Catalogue has been created only for the purposes of the demonstration. There is NO ASSURANCE OF QUALITY and therefore the catalogue should NOT be used for any other purposes than demonstration!}

\begin{figure}[htbp]
	\centering
		\includegraphics[height=16cm, keepaspectratio=true]{./figures/Subsahara_Catalogue_GCMT_1.eps}
	\caption{Sub-Saharan Catalogue - GCMT}
	\label{fig:Subsahara_Catalogue_GCMT_1}
\end{figure}

\begin{figure}[htbp]
	\centering
		\includegraphics[height=16cm, keepaspectratio=true]{./figures/Subsahara_Catalogue_ISC_1.eps}
	\caption{Sub-Saharan Catalogue - ''New Homogenised''}
	\label{fig:Subsahara_Catalogue_ISC_1}
\end{figure}


\cleardoublepage
